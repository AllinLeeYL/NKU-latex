\newpage
% Chinese abstract
\phantomsection\addcontentsline{toc}{section}{摘要}\tolerance=500 % make abstract in TOC
\begin{abstract}
近年来,使用自然语言访问结构化数据成为了研究热点,其中将自然语言翻译成特定的SQL语句的语义解析过程是结构化数据的自然语言接口技术研究的核心。尤其是深度神经网络的应用,使得许多NLP任务在测试集中甚至达到了超越人类的准确性。我提出了一种基于统计的分析方法,将训练集中与聚合运算符强相关的单词编码成额外的向量信息输入解码层模型,来增强聚合操作运算符的预测准确率,我把该方法称作Aggregation Strongly Correlated Token Enhanced Prediction(ASCTEP)以方便后续说明。
\end{abstract}

{\noindent{\CJKfamily{zhhei}关键字}:统计方法;结构化数据;聚合运算符;自然语言接口}

% English abstract
\newpage
\phantomsection\addcontentsline{toc}{section}{Abstract}\tolerance=500 % make abstract in TOC
\renewcommand{\abstractname}{\fontsize{14}{21}\textbf{Abstract}}
\begin{abstract}
Recently, the use of natural language to access structured data has become a hot topic, and the semantic parsing process of translating natural language into specific SQL statements is the core of the research of natural language interface technology for structured data. In particular, the application of deep neural networks has made many NLP tasks more accurate than humans. I proposed a statistical-based analysis method that encodes the words which are strongly correlated with the aggregation operator in the training set into an additional vector information. Then this vector is inputed into decoding layer model to enhance the prediction accuracy of the aggregation operation operator, which I call Aggregation Strongly Correlated Token Enhanced Prediction (ASCTEP) for subsequent explanation.
\end{abstract}

{\noindent\textbf{Keywords}: Statistical Method; Structured Data; Aggregation Operator; Natural Language Interface}