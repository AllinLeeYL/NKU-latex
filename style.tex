\documentclass[a4paper, 12pt]{article}

\usepackage[hidelinks]{hyperref} % hpyerlink
% \usepackage[paper=a4paper, left=2cm, right=2cm, top=3cm, bottom=3cm]{geometry}
\usepackage[paper=a4paper, left=3.18cm, right=3.18cm, top=2.54cm, bottom=2.54cm, foot=1.75cm, head=1.5cm]{geometry}
\usepackage[UTF8]{ctex} % Chinese support
\usepackage[divpsnames, svgnames, x11names]{xcolor} % color
\usepackage{listings} % code
\usepackage{graphicx} % graph
\usepackage{fancyhdr} % head and foot
\usepackage{enumitem} % list
\usepackage{booktabs} % table with three lines
\usepackage{amsmath} % math equation
\usepackage{titletoc} % 目录格式
\usepackage{titlesec} % title format
\usepackage{pdfpages} % 引用pdf
\usepackage{fontspec} % 字体
\usepackage{setspace} % 行距
\usepackage{abstract} % abstract
\usepackage{caption} % caption
\usepackage{zhnumber} % Chinese number
\usepackage{gbt7714} % reference

\newcommand{\emphasize}[1]{\emph{#1}} % emphasize
\newcommand{\keyword}[1]{\textbf{#1}} % keyword
\newcommand{\counterclear}{\setcounter{equation}{0} \setcounter{figure}{0}}
% code 
\newcommand{\refcode}[1]{Code \ref{#1}} % code reference
% \renewcommand{\thelstlisting}{\arabic{section}.\arabic{lstlisting}}
% equation
\newcommand{\refeq}[1]{式(\ref{#1})} % equation reference
\renewcommand{\theequation}{\arabic{section}.\arabic{equation}}
% figure
\newcommand{\reffig}[1]{图\ref{#1}} % graph reference
\renewcommand{\thefigure}{\arabic{section}.\arabic{figure}}
% table
\newcommand{\reftab}[1]{表\ref{#1}} % table referenece
\renewcommand{\thetable}{\arabic{section}.\arabic{table}}

% abstract
\setlength{\absleftindent}{0pt}
\setlength{\absrightindent}{0pt}
% font
\setmainfont{Times New Roman}
\setCJKmainfont{SimSun}
\ctexset{
	abstractname={\fontsize{14}{21}\CJKfamily{zhhei}摘\quad 要},
	contentsname={\fontsize{16}{24}\CJKfamily{zhhei}目\quad 录},
	bibname={参考文献}}
% format of list
\setitemize{topsep=0em, itemindent=2em, itemsep=0em}
\setenumerate{topsep=0em, itemindent=2em, itemsep=0em}
% format of caption
\captionsetup{labelsep=space}
% format of TOC
\renewcommand{\thesection}{第\zhnum{section}章}
\renewcommand{\thesubsection}{第\zhnum{subsection}节}
\titlecontents{section}[4.3em]{\fontsize{15}{22.5}}{\contentslabel{4.3em}}{\hspace*{-4.3em}}{\titlerule*[1pc]{.}\contentspage}
\titlecontents{subsection}[6.3em]{\fontsize{14}{21}}{\contentslabel{4.3em}}{\hspace*{-6.3em}}{\titlerule*[1pc]{.}\contentspage}
% title format
\titleformat{\section}{\centering\bfseries\fontsize{15}{23}}{第\zhnum{section}章 \quad}{0em}{}
\titleformat{\subsection}{\centering\bfseries\fontsize{14}{21}}{第\zhnum{subsection}节 \quad}{0em}{}
% header and foot
\fancyhf{}
\fancyfoot[C]{\fontsize{9}{13.5}\thepage}
\pagestyle{fancy}
% format of code
\renewcommand{\lstlistingname}{代码}
\lstset{
	language=C,
	otherkeywords={make\_EHelper, EMPTY, make\_DHelper, uint8\_t, uint16\_t, uint32\_t, uint64\_t, vaddr\_t, rtlreg\_t, uintptr\_t, mode\_t, off\_t, size\_t},
	basicstyle=\small,
	keywordstyle=\color[RGB]{138, 41, 1}\bfseries,
	identifierstyle=,
	tabsize=4,
	commentstyle=\color{gray},
	backgroundcolor=\color[RGB]{247, 247, 247},
	stringstyle=\ttfamily,
	breaklines=true, % auto break line in case of overflow
	frame=trbl,
	numbers=left,
	numberstyle=\tiny\color{gray},
	backgroundcolor=\color{white}
}